\thispagestyle{plain} % Remove header
\addcontentsline{toc}{chapter}{General Introduction}
\section*{General Introduction}

In the rapidly changing world of today, digitalization has become a powerful trend behind innovation and economic growth.
Digitalization is the process of integrating various aspects of society from business processes to an everyday manner into digital technologies. In turn, this facilitated the emergence of new startups.

The term startup originates from the area of technological development, Silicon Valley, which has been significantly influencing the technological process worldwide since the 1970s due to the highly concentrated number of technology companies surrounding Stanford University. Initially, the companies in the area were related to semiconductor manufacturing; however, during the dot-com boom in the late 1990s, the technologies shifted to the Internet.

In contrast, startups have also thrived in the 2000s during which multiple Facebooks, Ubers, Airbnbs, and Teslas were born and grew to billion-dollar businesses. These multi-billion companies have also changed the world; however, their impact derives from creating new business models with the help of technology.

For example, the world’s largest media company Facebook does not create its media. 
The world’s largest accommodation provider, Airbnb, has no hotels, while the world’s leading taxi service, Uber, does not own any cars.

Currently, startups are a global phenomenon: a similar hype currently exists worldwide.
In addition to globally notable examples, there are also local initiatives to promote entrepreneurship: for instance, in Finland, there is an Aalto Entrepreneurship Society, which aims at bringing an excellent entrepreneurial spirit for all students.

Moreover, the total number of startups globally indicates the importance of digitalization as a transformational process that opens numerous opportunities for entrepreneurship and economic growth in many countries.

This report will give an overview of the AerMax Company, a new startup service featuring digitalization and the sustainable impact on high-altitude project management. The workflow for Project Managers, and Workers is streamlined and optimized to ensure success.
An easy-to-use platform, task assignments and progress and real-time tracking, message sending, and performance analytic is ensured by the Customer. 

\newpage
